Recognizing the objects is the most fundamental function for human to understand the world, the whole procedure of which only takes a few tens of milliseconds for human brain.
Recently, Convolutional Neural Network (CNN) shows its potential to replace the human engineered features, such as SIFT \cite{lowe1999object}, SURF \cite{bay2006surf} and HOG \cite{dalal2005histograms} etc, in the large object recognition tasks\cite{krizhevsky2012imagenet}\cite{zeiler2014visualizing}\cite{simonyan2014very}. In ILSVRC2014, the 1st prize winner, referred as GoogLeNet, achieved a 93.33\% top-5 accuracy, almost as good as human annotation\cite{szegedy2014going}.
Unlike the local features such as SIFT or SURF, which present an shallow interpretation of spatial property, deep CNN can automatically learn top-down hierarchical feature representations. Therefore, deep CNN has been intensively used as the feature extractor for image recognition\cite{farabet2013learning}.
Training a large deep CNN on real recognition problem is always a complicated task. The model contains hundreds of millions of parameters to learn and lots of hyper-parameters that can affect its performance. However, continuingly expending web-based datasets such as ImageNet promises the researchers to utilize large amount of labeled images and train very deep CNNs. The truth that deep CNN outperforms other shallow models by a large margin in some real image recognition tasks encourages researchers to build deeper architecture with powerful high performance hardware and larger datasets.

Even though, these large scale web-based datasets contain almost any desired categories, it is still not possible for them to include images across all the domains that people may interest.
Domain adaptation aims to build classifiers that are robust to mismatched data distributions. However, image recognition models trained from source domain are inherently biased due to the datasets. Previous empirical and theoretical studies have shown that the testing error is positively proportional to the difference between the test and training input distribution\cite{ben2007analysis}\cite{blitzer2008learning}. Many domain adaptation methods have been proposed to solve this bias, but most of them are limited to features extracted from shallow models\cite{daume2009frustratingly}\cite{yang2007adapting}\cite{aytar2011tabula}.
Since deep CNN models are trained on these very large image datasets and can learn hierarchical features, they have strong generalization ability and can be applied in many other domains. Applying the pre-trained model from ImageNet dataset to other domains without taking advantage of domain adaptation shows some impressive results\cite{Chatfield14} \cite{zeiler2014visualizing}. Some work can be found for deep learning domain adaptation, but is limited to identical categories for the source and target domain\cite{hoffman2013one}\cite{NIPS2014_Zhou}. To our best knowledge, almost none of the previous domain adaptation studies the problem while the target categories are different from the source.

In this paper, we first train the feature extractor with fine-tuned deep CNN on two food datasets. Our fine-tuned models achieve the state-of-the-art performance on both datasets and show that deep CNN can learn discriminative representations for food recognition task. We find that previous domain adaptation methods suffer when the categories in source domain and target domain are different. 
%In this paper, we apply two kinds typical of deep CNN architecture, AlexNet and GoogLeNet, on a real specific recognition problem, food recognition, and discuss some observations in fine-tuning the existing CNN architectures on this problem. To our best knowledge, little work has done to discuss the architecture of GoogLeNet while the architecture of AlexNet has been widely studied and improved. Also, little work has been found to compare two deep CNNs with different architectures. By comparing some statistics of the weights and feature maps of these two architectures, we find that the $1\times 1$ small receptive field used in GoogLeNet can improve both computational and training efficiency which leads to the success of the whole architecture.
%Then, we conduct several experiments to stimulate a real world scenario when the training labeled data is rare. The results reveal that deep CNN can work well while transferring knowledge from general recognition task to specific one in this scenario. We achieve 95\% of the accuracy trained on full dataset while just utilizing half of the dataset.

%The rest of this paper is organized as follow: in Section \ref{exp}, the two deep CNN architectures and food image datasets are introduced. In Section \ref{discuss}, some experimental results are shown and we also compare the performance between the deep CNNs as well as some traditional methods on these two datasets. And some discussion of the Inception's architecture and statistics are shown in Section \ref{discuss}. We also show some further transfer learning results when the target training examples are rare for each class.
